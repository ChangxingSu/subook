
\chapter*{Perface}

I am of the opinion that every \LaTeX\xspace geek, at least once during 
his life, feels the need to create his or her own class: this is what 
happened to me and here is the result, which, however, should be seen as 
a work still in progress. Actually, this class is not completely 
original, but it is a blend of all the best ideas that I have found in a 
number of guides, tutorials, blogs and tex.stackexchange.com posts. In 
particular, the main ideas come from  four  sources:

\begin{itemize}
	\item The 
		\href{https://github.com/Tufte-LaTeX/tufte-latex}{Kaobook}, which was a model for the style.
	\item The \href{https://github.com/ElegantLaTeX/ElegantBook}{ElegantBook},which provided  elegant and colorful theorem  environment.
	\item \citetitle{thecell},which provides a very good typesetting style for biology textbooks, is one of the reference objects of this template.
	\item \href{https://github.com/Adhumunt/NotesTeX}{NotesTeX},which makes good use of the marginnote environment.
\end{itemize}

The first chapter of this book is introductory and covers the most
essential features of the class. Next, there is a bunch of chapters 
devoted to all the commands and environments that you may use in writing 
a book; in particular, it will be explained how to add notes, figures 
and tables, and references. The second part deals with the page layout 
and design, as well as additional features like coloured boxes and 
theorem environments.Subsequent chapters provide writing examples for different subject contents.


I started writing this class as an experiment, and as such it should be 
regarded. Since it has always been intended for my personal use, it may
not be perfect but I find it quite satisfactory for the use I want to 
make of it. I share this work in the hope that someone might find here 
the inspiration for writing his or her own class.

\begin{flushright}
	\textit{Changxing Su}
\end{flushright}
